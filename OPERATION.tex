\usepackage{hyperref}
\usepackage{longtable,tabularx,threeparttable}
\usepackage[dvipsnames,svgnames,table,hyperref]{xcolor}
\usepackage{textcomp}
\usepackage{url,listings}
\usepackage{fancybox}

\lstset{language=C++,
        basicstyle=\ttfamily\small,
        keywordstyle=\color{Maroon},
        commentstyle=\color{Blue}, 
        stringstyle=\color{Green},
        showstringspaces=false}

\begin{document}

\title{Operation Guide for Smart-Snmpd}

\author{Jens Rehsack}

\date{\today{}}

\maketitle

\tableofcontents

\section{Introduction}

\subsection{Audience}

This document targets people who have to administer smart-snmpd,
either because they operate machines with smart-snmpd running or they
have to configure reasonable default installations for packaging or
automated deployment.

Packagers should read the INSTALL document first.

\section{Operate a Smart-Snmpd Server}

\subsection{Set up and Run a Simple Smart-Snmpd}

\label{simple-setup}

The build procedure of smart-snmpd creates and installs an example
configuration file into \texttt{\$sysconfdir} (\texttt{\$prefix/etc})
named smart-snmpd.conf.example. You need to copy this file into a file
named smart-snmpd.conf and modify it to fit your requirements. The
task of copying this configuration file might be allocated to the
post-install script of a package.

The file \texttt{smart-snmpd.conf.example} is extensively documented; it is
strongly recommended you keep the comments when editing the configuration
file in daily business.

Note that unspecified values are set to reasonable defaults.

\begin{minipage}{\textwidth}
\subsubsection{Basic daemon settings}

These settings affect all (or at least multiple) components of the
\texttt{smart-snmpd}:

\begin{threeparttable}
\caption{Basic Settings}
\begin{tabularx}{\textwidth}{@{}*{2}{l}%
 >{\setlength\hsize{0.2\hsize}}X
 >{\setlength\hsize{0.5\hsize}}X@{}
}
\hline
\textbf{Setting} & \textbf{Data Type} & \textbf{Restriction} & \textbf{Description}\\
\hline
\texttt{daemonize} & \texttt{boolean} & - & When set true (default),
\texttt{smart-snmpd} will daemonize when started\\
\texttt{port} & \texttt{integer} & $ > 0 $ & Specifies the port to listen on, but
listen on all available interfaces / addresses\\
\texttt{listen-on} & \texttt{string} & must include port & Specifies the UDP
address to listen on (IPv4 or IPv6 address and port)\\
\texttt{status-file} & \texttt{string} & - & Specifies the fully qualified
path name (FQPN) to the file to store the status of the snmpd\\
\texttt{pid-file} & \texttt{string} & - & Specifies the FQPN to the
pid-file to use\\
\texttt{job-threads\tnote{1}} & \texttt{integer} & $ > 0 $ & Specifies the number
of threads which will be used to answer snmp requests \\
\texttt{rlimits} & \texttt{struct} & - & Provides settings for the (soft)
resource limits of the daemon for \texttt{core (RLIMIT\_CORE)},
\texttt{cpu (RLIMIT\_CPU)}, \texttt{data (RLIMIT\_DATA)},
\texttt{filesize (RLIMIT\_FSIZE)}, \texttt{files (RLIMIT\_NOFILE)},
\texttt{stack (RLIMIT\_STACK)} and \texttt{mem (RLIMIT\_AS)}.

Allowed values are: \texttt{integer} (specifying an absolute limit),
''\texttt{unlimited}'', ''\texttt{soft}'' (current set soft limit),
''\texttt{hard} (current set hard limit)''. If not specified, the
default is ''\texttt{soft}''.\\
\texttt{on-fatal\tnote{2}} & \texttt{enum} & - & defines the action on fatal
errors (\texttt{ERROR\_LOG} with log-level 0) - possible values:
\texttt{ignore}, \texttt{raise}, \texttt{kill} (default), \texttt{exit},
\texttt{\_exit}, \texttt{abort}\\
\texttt{su-cmd\tnote{3}} & \texttt{string} & - & Command to use to
execute an external command as a specific user\\
\texttt{su-args\tnote{3}} & \texttt{string} & - & Arguments for
above command\\
\hline
\end{tabularx}

\begin{tablenotes}
\item[1] only with threadpool support compiled into agent++\\
\item[2] only relevant when compiled with log4cplus (handler in snmp++ always uses raise(3) sending a \texttt{SIGTERM})
\begin{description}
\item[\texttt{ignore}] ignore the fatal error
\item[\texttt{raise}] send the signal \texttt{SIGTERM} using raise(3)
\item[\texttt{kill}] send the signal \texttt{SIGTERM} using kill(2) on getpid(2)
\item[\texttt{exit}] exits from the currently running process using exit(3), which means some cleanup tasks are done by the runtime library before the process is reaped by the kernel
\item[\texttt{\_exit}] exits from the currently running process using the \_exit(2) syscall, which means the process is reaped immediately
\item[\texttt{abort}] exits from the currently running process using abort(3), which means, a core file is written and after that the process is killed by a \texttt{SIGABRT} signal
\end{description}
\textbf{Note} that the last three actions might not flush all logging messages\\
\item[3] only with libjson (bundled or separate) and -\--with-su-cmd enabled in configuration
\end{tablenotes}

\end{threeparttable}
\end{minipage}

\begin{minipage}{\textwidth}
Example:
\begin{lstlisting}[language=C++,inputencoding=latin9,frame=shadowbox]
// listen on extra port
port = 8161

// default:
// status-file = /opt/smart-snmpd/var/db/smart-snmpd/status.db

su-cmd = "/usr/bin/sudo"
su-args = {"-u", "%u", "%c"}

// how many request-threads shall run
job-threads = 24

rlimits {
    core = "unlimited" // always write core dumps on SIGSEGV
    files = 512 // usually more than enough
}

on-fatal = _exit // rigorous exiting on fatal error
\end{lstlisting}
\end{minipage}

\begin{minipage}{\textwidth}
\subsubsection{Logging settings}

These settings affect the logging behavior of \texttt{smart-snmpd} and are
only available when logging support wasn't disabled in snmp++.

\begin{threeparttable}
\caption{Basic Settings}
\begin{tabularx}{\textwidth}{@{}*{2}{l}%
% >{\setlength\hsize{0.2\hsize}}X
 >{\setlength\hsize{0.5\hsize}}X@{}
}
\hline
\textbf{Setting} & \textbf{Data Type} & \textbf{Description}\\
\hline
\texttt{log-file\tnote{1}} & \texttt{string} & Specifies the FQPN of the file to write logs to\\
\texttt{log4cplus-property-file\tnote{2}} & \texttt{string} & Specifies the FQPN of the property file to configure log4cplus \\
\texttt{log-profile\tnote{3}} & \texttt{string} &
Specifies the log-profile to use. In addition to the 9 profiles provided by
snmp++, a profile named ''individual'' is provided which means ''Use the
individual log level configuration in \texttt{log-class}''.\\
\texttt{log-class} & \texttt{struct} & Defines each
each log-level per class (\texttt{error}, \texttt{warning},
\texttt{event}, \texttt{info}, \texttt{debug}, \texttt{user\tnote{4}}) individually.

Permitted values are \texttt{-1} for disabled and a threshold value from
\texttt{0} to \texttt{15}.\\
\hline
\end{tabularx}

\begin{tablenotes}
\item[1] only when compiled without log4cplus (using snmp++ logging builtin)\\
\item[2] only with log4cplus \\
\item[3] only with log-profiles enabled in snmp++\\
\item[4] snmp++ log level \texttt{user} is mapped to \emph{log4cplus'} log level \texttt{TRACE} when logging via log4cplus.
\end{tablenotes}

\end{threeparttable}
\end{minipage}

\begin{minipage}{\textwidth}
Example:
\begin{lstlisting}[language=C++,inputencoding=latin9,frame=shadowbox]
log4cplus-property-file = /opt/smart-snmpd/etc/log.properties

/**
 * log levels
 *
 * log classes:
 * - error
 * - warning
 * - event
 * - info
 * - debug
 * - user
 *
 * level can be from 0 .. 15 or -1 for off
 * all log entries with a log level lower or equal of the
 * configured will be shown
 *
 * defaults:
 * - 15 for error, warning (show all)
 * - 10 for event
 * -  5 for info
 * -  0 for debug, user (only most important)
 */

// reduce log level for less important log classes

log-class {
    event = 5
    info = 1
    debug = -1
}
\end{lstlisting}
\end{minipage}

\begin{minipage}{\textwidth}
\subsubsection{Statgrab settings}

These settings affect the wy statistics are collected with \textit{libstatgrab}.
Currently only file systems can be filtered.

\begin{threeparttable}
\caption{Statgrab Settings}
\begin{tabularx}{\textwidth}{@{}*{2}{l}%
% >{\setlength\hsize{0.2\hsize}}X
 >{\setlength\hsize{0.5\hsize}}X@{}
}
\hline
\textbf{Setting} & \textbf{Data Type} & \textbf{Description}\\
\hline
\texttt{valid-filesystems} & \texttt{list} & Specifies a list of file
system types to collect statistics for. With a ''!'' as first item
the specified list is subtracted from the initial list of valid
file systems\\
\hline
\end{tabularx}

\end{threeparttable}
\end{minipage}

\begin{minipage}{\textwidth}
Example:
\begin{lstlisting}[language=C++,inputencoding=latin9,frame=shadowbox]
statgrab {
    // filter all kind of networking file systems to
    // avoid hanging when the server isn't available
    valid-filesystems = { "!",
                          "nfs", "nfs3", "nfs4",
                          "cifs", "smbfs", "samba"
    }
}\end{lstlisting}
\end{minipage}

\begin{minipage}{\textwidth}
\subsubsection{Basic MIB Settings}

The built-in basic MIBs are setup using either the \emph{mibobject} or
\emph{inrobject}. The \emph{mibobject} structure initializes the basic
configuration settings of all managed \texttt{smart-snmpd} MIBs:

\begin{threeparttable}
\caption{Basic MIB Settings}
\begin{tabularx}{\textwidth}{@{}*{2}{l}%
 >{\setlength\hsize{0.5\hsize}}X@{}
}
\hline
\textbf{Setting} & \textbf{Data Type} & \textbf{Description}\\
\hline
\texttt{mib-enabled} & \texttt{boolean} & Specifies whether this MIB object
is enabled and can be requested or not. By default all built-in mibs are
enabled.\\
\texttt{async-update} & \texttt{boolean} & Specifies whether this MIB object
will be updated asynchronously in a background thread or synchronously
every time when requested. By default a MIB is updated synchronously.\\
\texttt{cache-timeout} & \texttt{integer} & Time in seconds before the
data cache of a MIB object becomes invalid and must be refreshed. This
happens all 30 seconds by default.\\
\hline
\end{tabularx}
\end{threeparttable}
\end{minipage}

\begin{minipage}{\textwidth}
Example:
\begin{lstlisting}[language=C++,inputencoding=latin9,frame=shadowbox]
mibobject DaemonStatus {
    // mib-enabled = false
    async-update = true
    cache-timeout = 60 /* update once a minute */
}
\end{lstlisting}
\end{minipage}

\begin{minipage}{\textwidth}
\subsubsection{Interval MIB Settings}

Several MIBs support differences, too. Those MIBs can be configured
using the \emph{inrobject} which contains all settings of \emph{mibobject}
and extends it with:

\begin{threeparttable}
\caption{Interval MIB Settings}
\begin{tabularx}{\textwidth}{@{}*{2}{l}%
 >{\setlength\hsize{0.5\hsize}}X@{}
}
\hline
\textbf{Setting} & \textbf{Data Type} & \textbf{Description}\\
\hline
\texttt{mr-interval} & \texttt{integer} & Time in seconds of the most-recent
interval to calculate the difference. The value must be a multiple of
\texttt{cache-timeout}. \textbf{Note:} If no interval is given, the value
of \texttt{cache-timeout} is used which results in a diff between the most
current value set and the one before.\\
\hline
\end{tabularx}
\end{threeparttable}

It is strongly recommended you configure a MIB object which delivers
differences to be updated asynchronously via a background
thread. Without this, it is not guaranteed that the difference time slices are
valid. In the case of a synchronous configuration, \texttt{smart-snmpd}
will issue out a warning.

If you do not configure eg. CpuUsage using the inrobject section, but
mibobject, the most recent cache interval is by default set to the value of
cache-timeout. This means at least the difference from the last statistics
is kept.
\end{minipage}

\begin{minipage}{\textwidth}
Example:
\begin{lstlisting}[language=C++,inputencoding=latin9,frame=shadowbox]
inrobject CpuUsage {
    async-update = true // strongly recommended for
                        // interval monitoring
    cache-timeout = 30  // also: measure interval base
    mr-interval = 600   // calculation time difference interval
}
\end{lstlisting}
\end{minipage}

\begin{minipage}{\textwidth}
\subsubsection{Setting up External Command MIBs}

\texttt{smart-snmpd} supports MIBs from external commands in addition to the
planned \emph{AgentX} support. Those external command MIBs are designed for
performance - the \emph{AgentX} protocol slows down mass data retrieves.

The external command MIBs are below \texttt{1.3.6.1.4.1.36539.20} and
need to be specified for each tree separately in the configuration using
the \emph{extobject} group, which extends the \emph{mibobject} specification
as follows:

\begin{threeparttable}
\caption{External Command MIB Settings}
\begin{tabularx}{\textwidth}{@{}*{2}{l}%
 >{\setlength\hsize{0.5\hsize}}X@{}
}
\hline
\textbf{Setting} & \textbf{Data Type} & \textbf{Description}\\
\hline
\texttt{command} & \texttt{string} & Specifies the FQPN to the command to
execute and the full arguments as a shell expression\\
\texttt{args} & \texttt{list of strings} & Specifies the arguments passed
to the executed command\\
\texttt{user\tnote{1}} & \texttt{string} & Specifies the name of the user to switch
to for command execution.\\
\texttt{sub-oid} & \texttt{integer} & specifies the oid below
\texttt{1.3.6.1.4.1.36539.20} where the received data should be addressed.\\
\texttt{rlimits} & \texttt{struct} & Allows setting the current (soft)
resource limits for the executed process for \texttt{core (RLIMIT\_CORE)},
\texttt{cpu (RLIMIT\_CPU)}, \texttt{data (RLIMIT\_DATA)},
\texttt{filesize (RLIMIT\_FSIZE)}, \texttt{files (RLIMIT\_NOFILE)},
\texttt{stack (RLIMIT\_STACK)} and \texttt{mem (RLIMIT\_AS)} individually.

Allowed values are: absolute limit (\texttt{integer}), \texttt{unlimited},
\texttt{soft} (soft limit at daemon start), \texttt{hard} (hard limit at daemon start).\\
\hline
\end{tabularx}
\begin{tablenotes}
\item[1] only with -\--with-su-cmd enabled in configuration
\end{tablenotes}
\end{threeparttable}

If you configure a MIB for an external object using mibobject instead of
extobject, no external command will be known and the MIB will remain
empty.
\end{minipage}

\begin{minipage}{\textwidth}
Example:
\begin{lstlisting}[language=C++,inputencoding=latin9,frame=shadowbox]
extobject AppMonitoring {
    async-update = true
    cache-timeout = 600 # 10 minutes
    command = /opt/smart-snmpd/libexec/plugin/appmon
    args = { "--output", "json" }
    user = "" // must be defined even if no user-switch is wanted
    sub-oid = 1
    rlimits {
        core = "soft"
        files = "soft"
    }
}\end{lstlisting}
\end{minipage}

\begin{minipage}{\textwidth}
\subsubsection{Built-In Mibs and their Names}

The following built-in MIB objects need settings from \ldots

\begin{threeparttable}
\caption{Settings for built-in MIBs}
\begin{tabular}{@{}*{3}{l}@{}}
\hline
\textbf{MIB Object} & \textbf{OID} & \textbf{Setting Structure}\\
\hline
\texttt{DaemonStatus} & \texttt{1.3.6.1.4.1.36539.10.1} & \emph{mibobject}\\
\texttt{HostInfo} & \texttt{1.3.6.1.4.1.36539.10.2} & \emph{mibobject}\\
\texttt{CpuUsage} & \texttt{1.3.6.1.4.1.36539.10.3} & \emph{inrobject}\\
\texttt{MemoryUsage} & \texttt{1.3.6.1.4.1.36539.10.4} & \emph{mibobject}\\
\texttt{SystemLoad} & \texttt{1.3.6.1.4.1.36539.10.5} & \emph{mibobject}\\
\texttt{UserLogins} & \texttt{1.3.6.1.4.1.36539.10.6} & \emph{mibobject}\\
\texttt{ProcessStatus} & \texttt{1.3.6.1.4.1.36539.10.7} & \emph{mibobject}\\
\texttt{FileSystemUsage} & \texttt{1.3.6.1.4.1.36539.10.8} & \emph{mibobject}\\
\texttt{DiskIO} & \texttt{1.3.6.1.4.1.36539.10.20} & \emph{inrobject}\\
\texttt{NetworkIO} & \texttt{1.3.6.1.4.1.36539.10.21} & \emph{inrobject}\\
\texttt{SwapIO} & \texttt{1.3.6.1.4.1.36539.10.22} & \emph{inrobject}\\
\texttt{AppMonitoring} & \texttt{1.3.6.1.4.1.36539.20.1} & \emph{extobject}\\
\hline
\end{tabular}
\end{threeparttable}
\end{minipage}

\begin{minipage}{\textwidth}
Memorize that
\begin{lstlisting}[language=C++,inputencoding=latin9,frame=shadowbox]
mibobject ProcessStatus {
    ...
}
\end{lstlisting}

is the same as

\begin{lstlisting}[language=C++,inputencoding=latin9,frame=shadowbox]
mibobject 1.3.6.1.4.1.36539.10.7 {
    ...
}
\end{lstlisting}
\end{minipage}

\subsubsection{Setting up External AgentX Satelites}

Future Task (Unimplemented)

\subsubsection{Setting up External MIBs in Shared Libraries}

Future Task (Unimplemented)

\subsubsection{Setting up Access Control}

\texttt{smart-snmpd} implements full \emph{SNMPv3} access control via embedded
\emph{agent++} library. For the specification see \url{http://www.ietf.org/rfc/rfc3415.txt}.


To configure \emph{SNMPv3 VACM}, the following structures should be used:

\begin{threeparttable}
\caption{\emph{user}: USM Configuration}

\begin{tabularx}{\textwidth}{@{}*{2}{l}
 >{\setlength\hsize{0.2\hsize}}X
 >{\setlength\hsize{0.6\hsize}}X@{}
}
\hline
\textbf{Setting} & \textbf{Data Type} & \textbf{Values} & \textbf{Description}\\
\hline
\texttt{auth-proto} & \texttt{enum} & \texttt{none}, \texttt{md5}, \texttt{sha} & hash algorithm for \texttt{auth-key}\\
\texttt{auth-key} & \texttt{string} & ~ & the authentication key\\
\texttt{priv-proto} & \texttt{enum} & \texttt{none}, \texttt{des},
\texttt{3des\tnote{*}}, \texttt{idea\tnote{*}}, \texttt{aes\tnote{1}},
\texttt{aes128}, \texttt{aes192\tnote{*}}, \texttt{aes256\tnote{*}}, & encryption algorithm for \texttt{priv-key}\\
\texttt{priv-key} & \texttt{string} & ~ & the private key\\
\hline
\end{tabularx}

\begin{tablenotes}
\item[*] not supported by all clients\\
\item[1] alias for aes128
\end{tablenotes}
\end{threeparttable}

The security name is taken from the section title of the individual user
sections specified in the configuration file.

\begin{threeparttable}
\caption{\emph{group}: Group Table Configuration}

\begin{tabularx}{\textwidth}{@{}*{2}{l}
 >{\setlength\hsize{0.2\hsize}}X
 >{\setlength\hsize{0.6\hsize}}X@{}
}
\hline
\textbf{Setting} & \textbf{Data Type} & \textbf{Values} & \textbf{Description}\\
\hline
\texttt{security-model} & \texttt{enum} & \texttt{usm} & Currently the one and only valid security model is USM\\
\texttt{security-name} & \texttt{list of strings} & Name of defined user & List of users (mapped to security name) valid for this group\\
\texttt{storage-type} & \texttt{enum} & \texttt{other}, \texttt{volatile}, \texttt{nonvolatile}, \texttt{permanent}, \texttt{readonly} & storage type for this entry\\
\hline
\end{tabularx}
\end{threeparttable}

The group name is taken from the section title of the individual group
sections specified in the configuration file.

\begin{threeparttable}
\caption{\emph{access}: View Table Configuration}

\begin{tabularx}{\textwidth}{@{}*{2}{l}
 >{\setlength\hsize{0.2\hsize}}X
 >{\setlength\hsize{0.6\hsize}}X@{}
}
\hline
\textbf{Setting} & \textbf{Data Type} & \textbf{Values} & \textbf{Description}\\
\hline
\texttt{sub-tree} & \texttt{OID} & OID\tnote{1} & Valid, managed OID or a parent of any\\
\texttt{mask} & \texttt{Hex Integer} & Bitmask as hexadecimal string &
See ViewTreeFamily description in RFC-3415\\
\texttt{view-type} & \texttt{enum} & \texttt{included}, \texttt{excluded} &
include or exclude this view tree\\
\texttt{storage-type} & \texttt{enum} & \texttt{other}, \texttt{volatile},
\texttt{nonvolatile}, \texttt{permanent}, \texttt{readonly} &
storage type for this entry\\
\hline
\end{tabularx}

\begin{tablenotes}
\item[1] \textbf{OID}: \textbf{O}bject \textbf{ID}entifier - also known
as a "MIB object identifier" or "MIB variable" in the SNMP network
management protocol. See \url{http://www.pcmag.com/encyclopedia\_term/0,2542,t=OID&i=48334,00.asp}
or \url{http://www.paessler.com/support/kb/questions/49} for details.
\end{tablenotes}
\end{threeparttable}

\begin{threeparttable}
\caption{\emph{access}: Access Table Configuration}

\begin{tabularx}{\textwidth}{@{}*{2}{l}
 >{\setlength\hsize{0.2\hsize}}X
 >{\setlength\hsize{0.6\hsize}}X@{}
}
\hline
\textbf{Setting} & \textbf{Data Type} & \textbf{Values} & \textbf{Description}\\
\hline
\texttt{group-name} & \texttt{string} & Name of defined group & Name of group whose access configuration is done here\\
\texttt{context} & \texttt{string} & & Context name for this configuration\\
\texttt{security-model} & \texttt{enum} & \texttt{usm} & Currently the one and only valid security model is USM\\
\texttt{security-level} & \texttt{enum} & \texttt{none}, \texttt{noauth}\tnote{1,2},
\texttt{nopriv}\tnote{1,2}, \texttt{auth}\tnote{1,2}, \texttt{priv}\tnote{1,2} &
Security level for this access entry\\
\texttt{match} & \texttt{enum} & & \ldots\\
\texttt{read-view} & \texttt{string / reference} & Name of defined view & Read view for this access entry\\
\texttt{write-view} & \texttt{string / reference} & Name of defined view & Write view for this access entry\\
\texttt{notify-view} & \texttt{string / reference} & Name of defined view & Notify view for this access entry\\
\texttt{storage-type} & \texttt{enum} & \texttt{other}, \texttt{volatile}, \texttt{nonvolatile}, \texttt{permanent}, \texttt{readonly} & storage type for this entry\\
\hline
\end{tabularx}

\begin{tablenotes}
\item[1] when not ''none'' used, the specified combination must be in the order of first auth/noauth and then priv/nopriv\\
\item[2] noauth,priv is an invalid combination
\end{tablenotes}
\end{threeparttable}

\pagebreak
\subsection{Invoking the Smart-Snmpd}

\subsubsection{Command Line Interaction}

The available command line arguments are displayed when \texttt{smart-snmpd}
is called using the ''-h'' switch. The one which will be discussed here in
detail is \texttt{-k}. Using the \texttt{-k} command line switch allows
you to choose the action to run from \texttt{start} (default), \texttt{stop},
\texttt{restart}, \texttt{graceful}, \texttt{reload}, \texttt{check},
\texttt{kill}.

\begin{threeparttable}
\caption{Actions on command line}

\begin{tabularx}{\textwidth}{@{}*{1}{l}
 >{\setlength\hsize{0.75\hsize}}X@{}
}
\hline
\textbf{Action Name} & \textbf{Action Description}\\
\hline
\texttt{start} & starts the \emph{Smart-SNMP-Daemon}\\
\texttt{stop} & Stops the currently running \emph{Smart-SNMP-Daemon}. Requests in the queue are answered (up to 2 seconds\tnote{*}).\\
\texttt{restart} & Restarts\tnote{1} the currently running \emph{Smart-SNMP-Daemon}. Requests in the queue aren't answered.\\
\texttt{graceful} & Gracefully restarts the currently running \emph{Smart-SNMP-Daemon}. Requests in the queue are answered (up to 2 seconds\tnote{*}).\\
\texttt{reload} & Reloads\tnote{2} the configuration of the currently running \emph{Smart-SNMP-Daemon}\\
\texttt{check} & Performs a check of the configuration file of the \emph{Smart-SNMP-Daemon}\\
\texttt{kill} & Kills\tnote{1} the currently running \emph{Smart-SNMP-Daemon}\\
\hline
\end{tabularx}

\begin{tablenotes}
\item[*] On small powered systems with heavy load a massive tailback of
requests can result in dropping requests even when restarting gracefully.
This is a limitation of UDP and cannot be fixed without forcing a denial
of service for new requests.\\
\item[1] sends a \texttt{SIGKILL} instead of the usually sent \texttt{SIGTERM}
which should force the operating system to remove the process.
\item[2] Reload (via sending the signal \texttt{SIGHUP} to the daemon process)
the configuration does neither affect the community/authentication entries nor
log4cplus configuration due locking bug in current versions\\
\end{tablenotes}
\end{threeparttable}

Runtime settings, which are refreshed when the smart-snmpd catches a
SIGHUP signal, are:

\begin{itemize}
\item{log-file}
\item{log-profile (when snmp++ is compiled with --with-log-profile)}
\item{log-class (for log-profile = ''individual'')}
\item{all mibobject, inrobject and extobject definitions}
\item{su-cmd and su-args}
\end{itemize}

\subsubsection{Overriding Configuration Settings}

In some circumstances it is reasonable to temporarily override some
configuration settings. You can do this from the command line when
starting the daemon (and only then) by specifying one or more of following
options:

\begin{description}
\item[-f]{loads config from specified file}
\item[-p]{uses specified pid-file}
\item[-s]{use specified status file}
\item[-l]{logs into specified file}
\item[-L]{uses specified log-profile}
\end{description}

An already running daemon cannot be forced to reload into the foreground (or
vice versa), you need to restart the daemon with appropriate \texttt{-d}
or \texttt{-D} switch.

\textbf{Note:} command line switches are valid until the daemon
ends or gets restarted with different command line switches. There is no
way to override such settings without stopping the daemon at least for a
moment.

\subsubsection{Requirements to start a Smart-Snmpd}

The \emph{Smart-Snmpd} has temporary and permanent requirements to operate.

The permanent requirements are:
\begin{itemize}
\item The directories to store the log-file, the status-file and the pid-file
must exist and the operating user must have write permissions to those
directories and the optional existing files from the last run.
\item The file system(s) containing above mentioned directories must have
enough free space to modify the named files.
\item No other process must operate on the configured UDP port
\item The system must have enough resources to start the configured
\texttt{job-threads}, the configured \texttt{async-threads}, the configured
thread for asynchronous logging and the thread for the signal handler.
\item Limits for the process must be large enough:
\begin{itemize}
\item at least 64MB virtual memory,
\item at least the permission to open up to 16 files - more when multiple external MIBs are configured
\item a stack size of at least 16KB
\end{itemize}
\end{itemize}

The temporary requirements are:
\begin{itemize}
\item The system must have enough resources (eg. free process table entries)
so the \texttt{smart-snmpd} process can be forked twice.
\item The system must have at least one available System V Semaphore.
\item The operating user must have read permissions to the specified
(either on command line or during build configuration) configuration file
\end{itemize}

\subsection{Run a Smart-Snmpd in the Build Environment}

\label{setup-build}

Command line arguments always override configuration file settings and
default behaviours as described in ''Setting up and Run a Simple
Smart-Snmpd'' (Section \ref{simple-setup}).

Usually you do not want to run a local smart-snmpd using elevated
privileges. That implies, the daemon can neither bind to port 161
nor modify \$prefix/var/run/smart-snmpd.pid nor any other required
status file (as the boot counter). It implies further, that some
statistics might be incomplete or unavailable, depending on the
underlying operating system.

Thus you need to override those file locations using
\begin{verbatim}
  $ ./src/smart-snmpd -f `pwd`/etc/smart-snmpd-test.conf \
                      -p `pwd`/smart-snmpd.pid \
                      -s `pwd`/smart-snmpd.db \
                      -l `pwd`/smart-snmpd.log
\end{verbatim}

\section{Platform Specific Issues}

\subsection{Windows NT}

Unimplemented.

\subsection{VMS}

Unimplemented.

\subsection{MacOS X}

Partial working.

\section{Language Bindings}

Future Task

\end{document}
